\section{Introduction}

\subsection{Note}

\emph{In this document, we write \enquote{Bulletproofs} to refer to the scheme
and \enquote{bulletproofs} to refer to the individual proof(s).}

\subsection{What Are Bulletproofs?}

Bulletproofs are a system to efficiently prove \emph{knowledge} of some value to another party in \emph{zero-knowledge}.
Often, an \emph{inner product argument} is used in the background.
Inner products can encode all sorts of things,
such as all $\NP$ languages (word problem) and all arithmetic circuits (circuit satisfiability).
This means that the statements that bulletproofs prove are (at least) $\NP$-hard,
which is enough power to encode practically all problems.

There are many upsides to using Bulletproofs as opposed to other schemes like KZG and FRI.
The name \enquote{Bulletproofs} means \enquote{short as a bullet with bulletproof security assumptions},
which hints at the logarithmic proof size and the conservative security assumptions.
%
Bulletproofs are shorter than other proofs (FRI) although there exist schemes with proofs of constant size (KZG).
Small proof size is perfect for blockchain applications,
where block space is extremely valuable.
%
Bulletproofs rely on the hardness of the \emph{discrete logarithm},
which is a well-understood problem and which already underlies the security of digital signatures in Bitcoin and other cryptocurrencies.
Other schemes assume pairings (KZG) or collision-resistant hashes (FRI),
which are new assumptions that need to be adopted first.
%
Bulletproofs are \emph{transparent},
which means that anyone can create a proof that anyone can verify, without prior communication.
There is no \emph{trusted setup} like in KZG,
where a shared secret (\emph{common reference string}) is created among all participants in a \emph{trusted setup ceremony},
which is used in all proofs.

One downside of bulletproofs is that they take linear time to verify,
as opposed to logarithmic time (FRI) or constant time (KZG).
This makes bulletproofs useless for applications that try to compress large computations (validity rollups).

There were many buzzwords in the past few paragraphs,
terms that lacked any definition.
I marked them in \emph{italics}.
As we move forward in this document,
we will define these terms one by one in logical order.
For now,
they should give you a general overview of Bulletproofs and enable you to do some online searches.

\subsection{What Is This Document?}

In this document we build a deep understanding of what bulletproofs are and how they work.
This write-up resulted from a reading group that I hosted,
where we read \emph{From Zero (Knowledge) to Bulletproofs},
which I highly recommend you read alongside the present document.
I learned a lot, but there were many points where I wanted to dive deeper, in particular the proofs.
In this document we try to explain everything from first principles, to the degree possible.
We assume some familiarity with elliptic curves and cryptographic schemes,
although this is explained, too.
Everything related to zero-knowledge proofs is explained in great detail,
including the foundations and intuition that are sometimes skimmed over.
Most importantly,
this is a technical write-up,
so if you are looking for a high-level view without maths, better leave now;
if you want to dive deeper, then welcome to paradise.

\subsection{Resources}

Here are some helpful resources that you should check out.
In particular,
you should read \emph{From Zero (Knowledge) to Bulletproofs} which this document builds upon.

\begin{itemize}
    \item \href{https://pages.cs.wisc.edu/~mkowalcz/628.pdf}{Intuition: \emph{The Strange Cave of Ali Baba}}
    \item \href{https://github.com/AdamISZ/from0k2bp}{Main: \emph{From Zero (Knowledge) to Bulletproofs}}
    \item \href{https://people.cs.georgetown.edu/jthaler/ProofsArgsAndZK.html}{More detail: \emph{Proofs, Argument, and Zero-Knowledge}}
    \item \href{https://vitalik.ca/general/2021/11/05/halo.html}{Vitalik explains Bulletproofs: \emph{Halo and more: exploring incremental verification and SNARKs without pairings}}
    \item \href{https://dankradfeist.de/ethereum/2021/07/27/inner-product-arguments.html}{Other perspective on Bulletproofs: \emph{Inner Product Arguments}}
    \item \href{https://explained-from-first-principles.com/number-theory/}{Refresher on number theory: \emph{Explained from first principles}}
    \item \href{https://github.com/matter-labs/awesome-zero-knowledge-proofs}{List of resources: \emph{Awesome zero knowledge proofs}}
\end{itemize}

\subsection{Elliptic Curves}

\emph{The maths starts here; you have been warned.
Check out the \href{https://explained-from-first-principles.com/number-theory/}{refresher on number theory} if any of the following seems fuzzy.}

An \emph{elliptic curve} $\curve$ is a set of two-dimensional \emph{points} $P = (x, y) \in \curve$
such that the coordinates $x, y \in \field$ stem from some field $\field$.
In our case,
we deal with the field $\field = \integers_p$ of integers modulo some number $p \in \naturals$.
The points on the curve satisfy the equation $y^2 = x^3 + ax + b$ for some coefficients $a, b \in \naturals$.

The curve's operation operation is the \emph{addition} $P + Q$ of two points $P, Q \in \curve$ on the curve.
\emph{(We use the additive notation).}
Point addition is \emph{associative} $P + (Q + R) = (P + Q) + R$
and \emph{commutative} $P + Q = Q + P$ for all points $P, Q, R \in \curve$.
%
There is an (additive) \emph{identity} $\infinity \in \curve$,
called the \emph{point at infinity},
which satisfies $\infinity + P = P + \infinity = P$ for all $P \in \curve$.
%
Furthermore,
every point $P \in \curve$ has an \emph{inverse} $-P \in \curve$ such that $P + (-P) = \infinity$.

We choose $\curve$ such that its order is a prime number $n \in \naturals$.
This means that $\curve$ is cyclic,
i.e., any point $P \in \curve \setminus \{ \infinity \}$
can be added onto itself $n$ times to reach all points in $\curve$ (exactly once)
in a cycle which arrives at $\infinity$.
(The loop is closed in step $n + 1$ where we arrive at $P$ again).
%
We call points that can generate the entire curve in this manner \emph{generators},
and in fact every point but $\infinity$ is a generator.
Usually,
we pick an arbitrary point $G \in \curve$ and call it \emph{the} generator of $\curve$.

There is a scalar field $\scalars$ that is isomorphic to the curve $\curve$.
These are the integers $\integers_n$ modulo $n$.
%
\emph{Scalar multiplication} $n \cdot P$ refers to the repeated addition of $P$ onto itself $n$ times,
for $n \in \scalars$ and $P \in \curve$.
We often write $nP$ instead of $n \cdot P$, for brevity.
%
Recalling the generators $G$ from the previous paragraph,
we can write $nG = \infinity$ for every generator $G \in \curve$.
Every point $P \in \curve$ is the result of repeated additions of a generator $G$
onto itself $i$ times: $P = iG$ for some $i \in \scalars$.
We call $i$ the \emph{discrete logarithm} of $P$ with respect to $G$,
which we may also write as $P / G = i$.
\emph{(This is division, but colloquially the operation is known as logarithm.)}
%
Scalar multiplication is homomorphic: $(a + b)P = (a + b)(iG) = (ai + bi)G = aP + bP$,
for all $a, b \in \scalars$ and all $P \in \curve$.
In other words, scalar multiplication \emph{distributes} over point addition.
%
In equations $aP = bP$,
where $P \in \curve \setminus \{ \infinity \}$ and $a, b \in \scalars$,
we may divide both sides by $P$ because $P$ is a generator.
Both $aP$ and $bP$ are points on $P$'s cycle through $\curve$ which has no repetitions.
Therefore, for $aP = bP$ to hold, $a = b$ must also hold.

There is no algorithm to efficiently compute the discrete logarithm of a random point.
A brute-force algorithm has to check all scalars $0 \leq i < n$
(in case of \href{https://en.bitcoin.it/wiki/Secp256k1}{secp256k1}, $n \approx 2^{256}$).
This is where the cryptographic security of elliptic curves comes from.
%
Known algorithms such as \href{https://en.wikipedia.org/wiki/Baby-step\_giant-step}{Baby-step giant-step} and
\href{https://en.wikipedia.org/wiki/Pollard\%27s\_rho\_algorithm\_for\_logarithms}{Pollard's rho algorithm}
compute the discrete log in time $\approx \mathcal{O}(\sqrt{n})$ which is much faster but still not polynomial.
